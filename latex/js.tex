% Full instructions available at:
% https://github.com/elauksap/focus-beamertheme

\documentclass{beamer}
\usetheme[numbering=none]{focus}
\usepackage[utf8x]{inputenc} %So spanish accents will work%
\usepackage{hyperref}
\usepackage{listings}
\usepackage{color}
\usepackage{verbatim}
\usepackage{inconsolata}
\usepackage{tikz-cd}

\definecolor{codegray}{rgb}{0.5,0.5,0.5}
\definecolor{codecyan}{RGB}{0, 255, 255}
\definecolor{codeyellow}{RGB}{240,219,78}

\lstdefinestyle{mystyle}{
    commentstyle=\color{codeyellow},
    keywordstyle=\color{magenta},
    numberstyle=\color{codegray},
    stringstyle=\color{codecyan},
    basicstyle=\ttfamily\small,
    breakatwhitespace=false,         
    breaklines=true,                 
    captionpos=b,                    
    keepspaces=true,
    showspaces=false,                
    showstringspaces=false,
    showtabs=false,                  
    tabsize=2
}

\lstdefinelanguage{JavaScript}{
keywords={typeof, new, true, false, catch, function, return, null, catch, switch, var, let, if, in, while, do, else, case, break, default, for, of},
ndkeywords={class, export, boolean, throw, implements, import, this},
ndkeywordstyle=\color{green}\bfseries,
sensitive=false,
comment=[l]{//},
morecomment=[s]{/*}{*/},
morestring=[b]',
morestring=[b]",
morestring=[b]`,
}
\lstset{style=mystyle, escapeinside=||}
\title{JavaScript}
\subtitle{Hour of Code 2018}

\titlegraphic{\includegraphics[scale=0.13]{images/js_logo.png}}
\institute{Universidad de Valladolid}
\date{27 Oct 2018}

\begin{document}
\setbeamertemplate{caption}{\raggedright\insertcaption\par}
\begin{frame}
        \maketitle
\end{frame}
    % PARAGRAPH SPACING. -----------------------------------------------------------------
    \setlength{\parskip}{\baselineskip}%
    \setlength{\parindent}{0pt}%

\begin{frame}{Presentación}
\centering
{\Large Carlos Gómez (@Kurolox)

Pablo Valdunciel (@pabvald)\bigskip\pause

 Link a la presentación:}

\centering\textbf{\hyperlink{https://github.com/Kurolox/Vue-HoC-18}{\Large github.com/Kurolox/Vue-HoC-18}}
\end{frame}
\begin{frame}{Qué es JavaScript}
    	\pause
        JavaScript (también llamado JS) es un lenguaje de programación originado en 1995.\pause

    	Es conocido principalmente por su uso en desarrollo web, permitiendo que un usuario pueda interaccionar con una pagina web.
\end{frame}
\begin{frame}{Qué es JavaScript}
        JavaScript es un lenguaje... \bigskip
        \begin{itemize}
            \item interpretado\pause
            \item que usa tipado débil\pause
            \item dinámico
    \end{itemize}
\end{frame}
    
\begin{frame}{JavaScript es interpretado}
        \pause
        Un lenguaje \textbf{interpretado} es aquel que es traducido a código máquina a medida que se ejecuta, mientras que un lenguaje \textbf{compilado} es aquel que se traduce a código máquina antes de ejecutarse.\pause
        \centering
        
        JavaScript es un lenguaje interpretado.
\end{frame}
    
\begin{frame}{JavaScript usa tipado débil}
        \pause
        Un lenguaje que usa \textbf{tipado débil} no necesita declarar el tipo de sus variables explícitamente, mientras que un lenguaje que usa \textbf{tipado fuerte} necesita declarar el tipo que va a almacenar una variable de antemano.\pause
        \centering
        
        JavaScript utiliza tipado débil.
\end{frame}
    
    
    
\begin{frame}{JavaScript es dinámico}
        \pause
        Un lenguaje \textbf{dinámico} comprueba los tipos durante la ejecución del programa, mientras que un lenguaje \textbf{estático} los comprueba antes de la ejecución.\pause
        \centering
        
        JavaScript es un lenguaje dinámico.
\end{frame}
\begin{frame}[fragile]{C: Compilado, tipado fuerte y estático}\pause
        \lstinputlisting[language=C++]{code_snippets/example.c}\centering
        
        Este programa ha de ser compilado con un compilador.
\end{frame}

\begin{frame}[fragile]{El mismo ejemplo, con JavaScript}\pause
        \lstinputlisting[language=JavaScript]{code_snippets/example.js}\centering
\end{frame}

\begin{frame}{Cómo ejecutar JavaScript}\pause
    Hay varias maneras de ejecutar JavaScript. La mas accesible es desde vuestro navegador de preferencia (Firefox, Chrome...)\pause
    
    Abrid las herramientas de desarrollador (F12), y buscad la opción de consola.
    \begin{figure}
        \centering
        \includegraphics[width=0.75\textwidth]{images/firefox_dev.png}
    \end{figure}
\end{frame}

\begin{frame}{Como ejecutar JavaScript}
Para este taller, vamos a utilizar JS Bin, un editor online de JavaScript.
    
\centering\url{https://jsbin.com/?js,console}
\end{frame}

\begin{frame}{ECMAScript}
Antes de empezar con JavaScript, deberíais saber esto...

ECMAScript es un estándar de lenguajes de programación, y JavaScript es una implementación de dicho estándar. \pause

ECMAScript se va actualizando con el tiempo con nuevas funcionalidades, y versiones antiguas de algunos navegadores pueden no estar actualizadas para soportar nuevas versiones de ECMAScript.
\end{frame}

\begin{frame}{ECMAScript}
En este taller, todo lo que os enseñe esta basado en ECMAScript 6 (También llamado ECMAScript 2015, o ES6) y esta soportado por todos los navegadores actuales. \pause

En caso de que tengáis que soportar navegadores antiguos (Internet Explorer 11 y menor, por ejemplo), algunas partes de este taller no funcionarán.
\end{frame}

\begin{frame}[fragile]{Variables y constantes}
Hay varias maneras de crear variables en JavaScript:
\begin{itemize}
    \item {\verb|var|} para declarar una variable \pause
    \item {\verb|let|} para declarar una variable \pause
    \item {\verb|const|} para declarar una constante
\end{itemize}

\end{frame}

\begin{frame}[fragile]{Var y let}
La diferencia entre {\verb|var|} y {\verb|let|} es simple. \pause
\begin{itemize}
    \item {\verb|var|} define las variables a un nivel funcional (pertenecen a la función en la que se definen o, en caso de que no pertenezcan a ninguna, son globales)
    \item {\verb|let|} define las funciones a un nivel de bloque (pertenecen al bloque en el que se definen o, en caso de que no pertenezcan a ninguno, son globales)\pause

\end{itemize}

Mostraremos ejemplos de esto mas adelante, cuando entendáis la sintaxis de JavaScript. En caso de duda, utilizad {\verb|let|} para evitar posibles problemas.
\end{frame}

\begin{frame}[fragile]{Const}
{\verb|const|} se utiliza para definir constantes. Esto implica que en la mayoría de los casos, una variable constante: \pause
\begin{itemize}
    \item no puede ser reasignada \pause
    \item no puede ser modificada* \pause
\end{itemize}

Intentar hacer cualquiera de estas cosas a una constante va a resultar en un error.
\end{frame}

\begin{frame}{Tipos de datos}
En JavaScript, tenemos los siguientes tipos de datos:\bigskip\pause

\begin{columns}[t, onlytextwidth]
            \column{0.5\textwidth}
                \textbf{Primitivos:}
                \begin{itemize}
                    \item Number
                    \item String
                    \item Boolean
                    \item Null
                    \item Undefined
                    \item Symbol
                \end{itemize} \pause
            
            \column{0.5\textwidth}
                \textbf{No primitivos:}
                \begin{itemize}
                    \item Object
                \end{itemize}
        \end{columns}
\end{frame}

\begin{frame}{Numbers}
Un \textbf{number} (Numero) es un tipo de dato numérico. \pause

A diferencia de otros lenguajes, no hay distintos tipos dependiendo de si el numero es decimal o no, o de cuanto espacio se quiera reservar (short, int, long, double, float...) \pause

En JavaScript, solo hay un único tipo, number. Este es similar a un double en otros lenguajes de programación (64 bits, punto flotante)
\end{frame}

\begin{frame}[fragile]{Numbers}
\begin{lstlisting}[language=JavaScript]
let meaningOfLife = 42;
typeof(meaningOfLife); // "number"|\pause|

let floatingPoint = 3.141592;
typeof(floatingPoint); // "number"|\pause|

// Tambien puedes utilizar notacion cientifica
let exponentialNotation = 123e5; // 123 * (10^5) = 12300000|\pause|

// O hexadecimal (0x) y octal (0)
let hexadecimalNum = 0xFF; // 255
let octalNum = 0147 // 103
\end{lstlisting}    
\end{frame}

\begin{frame}[fragile]{Numbers: Casos especiales}
\begin{lstlisting}[language=JavaScript]
// Y que pasa si hago esto?
let divisionByZero = 1/0;|\pause| // Infinity|\pause|
let negativeDivisionByZero = -1/0; // -Infinity|\pause|

// Y un numero extremadamente grande?|\pause|
let giganticNumber = 1e9999; // Infinity
let giganticNegativeNumber = -1e9999; // -Infinity|\pause|

// Infinity e -Infinity son de tipo number!
typeof(Infinity); // "number"
typeof(-Infinity); // "number"|\pause|

// Y esto de aqui?
let divisionByTomato = 4 / "tomato";|\pause| // "NaN"
\end{lstlisting}    
\end{frame}

\begin{frame}{Numbers: Casos especiales}
NaN es un número especial. Como habréis notado, JavaScript no suele devolver errores ante casos extraños.

Ya entraremos en detalle acerca de como JavaScript maneja estos casos, pero por ahora centrémonos en NaN.\pause

NaN (Not a Number; No es un número) es un valor numérico que JavaScript devuelve siempre que no encuentre un valor legal en una operación aritmética (como una string, con excepciones que ya mencionaremos)
\end{frame}

\begin{frame}[fragile]{Numbers: Operaciones}
\begin{lstlisting}[language=JavaScript]
let num1 = 9, num2 = 3;

// Operador de suma: +
let sumNums = num1 + num2; // 12

// Operador de resta: -
let substractNums = num1 - num2; // 6

// Operador de producto: *
let multiplyNums = num1 * num2; // 27

// Operador de division: /
let divideNums = num1 / num2; // 3

// Operador de resto: %
let reminderOfNums = num1 % num2; // 0
\end{lstlisting}
\end{frame}

\begin{frame}[fragile]{Numbers: Operaciones}
\begin{lstlisting}[language=JavaScript]
// Operador de incremento: ++
num1++; // num1 = 10;

// Operador de decremento: --
num2--; // num2 = 2;
\end{lstlisting}\pause

\begin{block}{Cuidado! Operaciones con NaN}
Cualquier operación que tenga NaN en alguno de sus operandos devolverá NaN. 
\begin{lstlisting}
let someOperation = 46 + (38 % NaN) * 27; // NaN
\end{lstlisting}
\end{block}
\end{frame}

\begin{frame}[fragile]{Numbers: Punto flotante}
Recordad que en JavaScript, \textbf{todos} los números son de punto flotante. Esto puede dar lugar a imprecisiones, y debéis de tener cuidado con estas.

\begin{lstlisting}[language=JavaScript]
0.1 + 0.2; // 0.30000000000000004

0.1 + 0.2 == 0.3; // false
\end{lstlisting}
\end{frame}

\begin{frame}[fragile]{Strings}
Un \textbf{String} es una cadena de caracteres usada para representar texto.

Puedes usar single quotes ('), double quotes (") o backticks (`) para delimitar un string, aunque estos últimos son un caso especial que ya mencionaremos.\pause

\begin{lstlisting}[language=JavaScript]
let singleQuotes = 'Soy una cadena de caracteres!';
let doubleQuotes = "Yo tambien!";
let backTicks = `Y yo!`;

typeof(singleQuotes); // "string"
\end{lstlisting}
\end{frame}

\begin{frame}[fragile]{Strings}
En caso de que queráis utilizar alguno de los caracteres delimitantes en una string, tenéis dos formas de hacerlo:\pause
\begin{itemize}
\item Utilizar un delimitador distinto: 
\begin{lstlisting}[language=JavaScript]
let doubleQuoteDelimited = "Uso ' y `!"

let singleQuoteDelimited = 'Y yo uso " y `!'

let backtickDelimited = `Uso " y ' sin problemas!`
\end{lstlisting}\pause
\item Escapar el caracter con \textbackslash:
\begin{lstlisting}[language=JavaScript]
let stringWithEscapedChar = "Esto \" funciona!"
\end{lstlisting}
\end{itemize}
\end{frame}

\begin{frame}[fragile]{Strings: Concatenación}
Las strings solo tienen un operador: concatenación (+)
\begin{lstlisting}[language=JavaScript]
let firstName = "Alberto";
let lastName = "Gonzalez";

let fullName = firstName + " " + lastName;
// fullName = "Alberto Gonzalez"
\end{lstlisting}\pause
\begin{block}{Cuidado! No confundáis los operadores}
Tanto el operador de concatenación como el de adición utilizan el mismo símbolo (+), pero son dos operadores completamente distintos. Ya entraremos en detalle acerca de esto mas adelante.
\end{block}
\end{frame}

\begin{frame}[fragile]{Strings: template strings}
En caso de que quieras tener un string con valores de variables, usar template strings puede ser muy útil para facilitar la tarea.\pause

Esto solo funciona si el string usa backticks como delimitador (`)\bigskip\pause

\begin{lstlisting}[language=JavaScript]
let firstName = "Alberto";
let lastName = "Gonzalez";

let fullName = `${firstName} ${lastName}`;
// fullName = "Alberto Gonzalez"
\end{lstlisting}
\end{frame}

\begin{frame}[fragile]{Boolean}
Un \textbf{Boolean} (Booleano) es un tipo de dato que sólo tiene dos valores: true (verdadero) o false (falso).\pause

\begin{lstlisting}[language=JavaScript]
let webDevRocks = true;
let javascriptIsGarbage = false;

typeof(webDevRocks); // "boolean"
typeof(javascriptIsGarbage); // "boolean"

\end{lstlisting}
\end{frame}

\begin{frame}{Boolean: Operadores booleanos}
Hay varios operadores booleanos: los operadores de comparación. Estos operadores comparan dos argumentos y siempre devuelven un booleano. \bigskip\pause

\begin{columns}[t, onlytextwidth]
            \column{0.5\textwidth}
                \textbf{Comparadores normales}
                \begin{itemize}
                    \item Igualdad (==)
                    \item No igualdad (!=)
                    \item Mayor que (>)
                    \item Menor que (<)
                    \item Mayor o igual que (>=)
                    \item Menor o igual que (<=)
                \end{itemize}\pause
            
            \column{0.5\textwidth}
                \textbf{Comparadores estrictos}
                \begin{itemize}
                    \item Igualdad (===)
                    \item No igualdad (!==)
                \end{itemize}
        \end{columns}
\end{frame}

\begin{frame}[fragile]{Boolean: Operadores booleanos}
\begin{lstlisting}[language=JavaScript]
4 == 7; // false
2 != 5; // true

3 > 4; // false
3 < 4; // true

2 > 2; // false
2 >= 2; // true
\end{lstlisting}
\end{frame}

\begin{frame}[fragile]{Boolean: Operadores booleanos}
JavaScript hace varias conversiones de tipos por detrás cuando trabajas con tipos distintos.
\begin{lstlisting}[language=JavaScript]
3 > "5" // 3 > 5, false
4 == "4" // 4 == 4, true
\end{lstlisting}\pause

Para evitar conversiones implícitas, utiliza los comparadores estrictos.
\begin{lstlisting}[language=JavaScript]
4 == "4" // 4 == 4, true
4 === "4" // false

2 != "2" // 2 != 2, false
2 !== "2" // true
\end{lstlisting}
Siempre que puedas, utiliza el comparador estricto para evitar problemas.


\end{frame}

\begin{frame}[fragile]{Null}
\textbf{null} es un tipo de dato que sólo tiene un valor: null.

null se usa para indicar que algo no tiene ningún valor; que esta vacío.

\begin{lstlisting}[language=JavaScript]
let void = null;

typeof(void); // "object" (por motivos de legacy)

\end{lstlisting}
\end{frame}

\begin{frame}[fragile]{Undefined}
\textbf{undefined} es un tipo de dato que solo tiene un valor: undefined.

Undefined se usa para indicar que algo no tiene ningún valor asignado.

\begin{lstlisting}[language=JavaScript]
let something;

typeof(something); // "undefined"
\end{lstlisting}
\end{frame}

\begin{frame}{Diferencia entre null y undefined}
null y undefined son dos cosas distintas, pese a parecer idénticas. Aquí tenéis un ejemplo visual.

    \begin{figure}
        \centering
        \includegraphics[width=0.7\textwidth]{images/undefinednull.png}
    \end{figure}
\end{frame}

\begin{frame}{Diferencia entre null y undefined}
\begin{itemize}
    \item Undefined indica que algo simplemente no esta ahí, y es el valor que toma cualquier variable por defecto.
    \item Null indica que el valor de algo es nulo, y has de asignarlo explícitamente a algo.
\end{itemize}
\end{frame}

\begin{frame}[fragile]{Conversiones implícitas}
JavaScript hace conversiones entre tipos por detrás cuando trabajas con tipos distintos.\pause

\begin{lstlisting}[language=JavaScript]
1 + "10" |\pause|// "110"|\pause|
"1" - 10 |\pause|// -9|\pause|
3 + ("Tomato" - "Pizza")  |\pause|// "3NaN"|\pause|
"b"+"a"+(+"a")+"a" |\pause|// "baNaNa"|\pause|
1 / 0 + " and beyond!" |\pause|// "Infinity and beyond!"
\end{lstlisting}
\end{frame}

\begin{frame}{Conversiones implícitas}
    \begin{figure}
        \centering
        \includegraphics[width=\textwidth]{images/excuse_me.jpg}
    \end{figure}
\end{frame}

\begin{frame}{Conversiones implícitas}
Si estás sumando algo:
\begin{itemize}
    \item Si alguno de los operandos es una string, el otro se convierte en string y se concatenan.
    \item En caso contrario, se intenta convertir todo a numero y sumar.
\end{itemize}

Si estas restando, multiplicando o dividiendo algo:
\begin{itemize}
    \item Se ntenta convertir todo a número (En caso de que no se pueda, se convierte en NaN)
\end{itemize}
\end{frame}

\begin{frame}{Tipos de datos no primitivos}
Sólo hay un tipo de dato no primitivo: El tipo Object (objeto)

Un object es una colección de propiedades. Se definen utilizando llaves: \{ y \}\pause

Cada propiedad tiene una "llave" (identificador) asignada, siguiendo el formato \textbf{\textit{llave : propiedad}}, y se separan con comas unos pares llave : propiedad de otros.
\end{frame}

\begin{frame}[fragile]{Objects}
\begin{lstlisting}[language=JavaScript]
market_stock = {"apples" : 4, "color" : "red"}

market_stock.apples // 4
market_stock.color // "red"
market_stock.precio // undefined

market_stock.apples-- // market_stock.apples = 3
market_stock.price = 4.95

market_stock // {apples: 3, color: "red", price: 4.95}
\end{lstlisting}
\end{frame}

\begin{frame}[fragile]{Arrays}

Un \textbf{array} es un tipo de objeto que te permite guardar múltiples valores distintos en una sola variable.

Una array se define con un par de corchetes [], separando los elementos con comas.

\begin{lstlisting}[language=JavaScript]
let someArray = [1, "manzana", 9, true];
typeof(someArray); // "object"
\end{lstlisting}


\end{frame}

\begin{frame}[fragile]{Arrays}
Puedes acceder a un elemento específico de un array especificando su índice (numeración basada en 0)

\begin{lstlisting}[language=JavaScript]
let someArray = [1, "manzana", 9, true];

someArray[0]; // 1
someArray[1]; // "manzana"
someArray[3]; // true
\end{lstlisting} 
\end{frame}


\begin{frame}{Estructuras de control}
JavaScript tiene las siguientes estructuras de control.
\begin{itemize}
    \item if / else if / else
    \item while / do while
    \item for
    \item switch
    \item for each
    \item with
\end{itemize}
\end{frame}

\begin{frame}{If / Else if / Else}
La sentencia \textbf{if} te permite ejecutar una parte del código sólo si una condición se cumple de antemano. \pause

Una condición es cualquier sentencia que devuelva un valor booleano, o pueda convertirse en uno.

Si no se cumple dicha condición, puedes comprobar por condiciones adicionales usando \textbf{else if}, o puedes ejecutar código si no se cumple ninguna condición usando \textbf{else}.
\end{frame}

\begin{frame}[fragile]{If / Else if / Else}
\begin{lstlisting}[language=JavaScript]
let someNumber = 5;

if (someNumber < 3) {
    console.log("el numero es menor que 3.");
} else if (someNumber < 10) {
    console.log("el numero esta entre 3 y 9.");
} else {
    console.log("el numero es mayor a 9.");
}
\end{lstlisting}
\end{frame}

\begin{frame}{While / Do while}
La sentencia \textbf{while} te permite repetir la ejecución de una parte del código mientras se cumpla una condición.\pause

La sintaxis de un bucle while es la siguiente:

\textbf{while (condicion) \{ código \} }\pause

Una sentencia \textbf{do... while} es idéntica a un while, pero esta garantiza que el código se ejecutara al menos una vez, incluso si la condición no se cumple la primera vez.

\textbf{do \{ código \} while (condicion) }

\end{frame}

\begin{frame}[fragile]{While / Do while}
\begin{lstlisting}[language=JavaScript]
let counter = 5;

while(counter < 13){
    console.log(counter++);
}
|\pause|
do {
    console.log("Condicion imposible!");
} while (counter < 0)   
\end{lstlisting}
\end{frame}

\begin{frame}{For}
La sentencia \textbf{for} te permite repetir la ejecución de una parte del código un numero determinado de veces.\pause

La sintaxis de un bucle for es la siguiente:

\textbf{for (contador; condición; paso) \{ código \} }\pause

El contador es cualquier variable que se use para llevar la cuenta del numero de iteraciones del bucle.

La condición es que se ha de cumplir para que el bucle siga repitiéndose: Cuando la condición deja de cumplirse, el bucle para.

El paso indica de que manera quieres modificar el contador tras cada iteración del bucle.
\end{frame}

\begin{frame}[fragile]{For}
\begin{lstlisting}[language=JavaScript]
for(counter = 10; counter > 0; counter--){
    console.log("Lanzamiento en " + counter);
}
console.log("Despegue!");
\end{lstlisting}
\end{frame}

\begin{frame}[fragile]{Switch}
La sentencia \textbf{switch} te permite evaluar una expresión, y ejecutar un determinado código en base a dicha expresión. \pause

La sintaxis de una sentencia switch es la siguiente:

\begin{lstlisting}[language=JavaScript]
switch (expresion) {
    case (condicion): 
        code
        break;
    case (condicion):
        code
        break;
    default:
        code
}\end{lstlisting}
\end{frame}

\begin{frame}[fragile]{Switch}
\begin{lstlisting}[language=JavaScript]
let fruit = "apple"
switch (fruit) {
    case banana: 
        console.log("Me quedan 3 platanos.")
        break;
    case apple:
        console.log("Me quedan 9 manzanas.")
        break;
    default:
        console.log("No me queda ninguna " + fruta)
}\end{lstlisting}
\end{frame}

\begin{frame}[fragile]{for each}
En JavaScript existen diferentes tipos de bucles \textbf{for each}, cada uno de los cuales tiene un comportamiento diferente: \pause
\begin{itemize}
     \item for...in
    \item for...of
   
\end{itemize}
\end{frame}


\begin{frame}[fragile]{for...in}
La sentencia \textbf{for...in} itera sobre todos los nombres de las propiedades de un objeto, en un \textbf{orden arbitrario}. Para cada uno de los elementos, se ejecuta la sentencia especificada. \pause

La sintaxis de un bucle \textbf{for...in} es:
\begin{lstlisting}[language=JavaScript]
for (variable in objeto) {
    code
}
\end{lstlisting}
\end{frame}

\begin{frame}[fragile]{for...in}
\begin{lstlisting}[language=JavaScript]
let prices = {"apple" : 1.25, "banana" : 2.30, "orange" : 4.5}; // Objeto

for (let fruit in prices) {
    console.log(fruit + " = " + prices[fruit]);
}

// banana = 2.30  <- Orden arbitrario
// apple = 2.30  
// orange = 4.5 


\end{lstlisting}
\end{frame}

\begin{frame}[fragile]{for...of}
La sentencia \textbf{for...of} crea un bucle que itera a través de los elementos de objetos iterables (incluyendo Array), ejecutando las sentencias de cada iteración con el valor del elemento que corresponda.\pause

La sintaxis de un bucle \textbf{for...of} es la siguiente:
\begin{lstlisting}[language=JavaScript]
for (variable of iterable) {
  code
}
\end{lstlisting}
\end{frame}

\begin{frame}[fragile]{for...of}
\begin{lstlisting}[language=JavaScript]
let iterable = [10, 20, 30]; //Array de numeros

for (let valor of iterable) {
  valor += 1;
  console.log(valor);
}
// 11
// 21
// 31
|\pause|
console.log(iterable); // El original no se modifica!
// [10, 20, 30]
\end{lstlisting}
\end{frame}

\begin{frame}[fragile]{Funciones}
Una \textbf{función} es una pieza de código que realiza una función determinada. \pause

Una función puede tener de 0 a múltiples parámetros de entrada, y puede devolver hasta un parámetro de salida.
\begin{lstlisting}[language=JavaScript]
function (entrada) {
    code;
    return salida;
}
\end{lstlisting}
\end{frame}

\begin{frame}[fragile]{Funciones}
Hay múltiples maneras de definir una función. Una de ellas es usando la palabra clave \textbf{function}:
\begin{lstlisting}[language=JavaScript]
function sumNumbers(num1, num2) {
    let total = num1 + num2;
    return total;
}\end{lstlisting} \pause
Otra manera es asignando la función a una variable o constante (se recomienda que sea una constante):
\begin{lstlisting}[language=JavaScript]
const sumNumbers = function (num1, num2) {
    let total = num1 + num2;
    return total;
}
\end{lstlisting}
\end{frame}

\begin{frame}[fragile]{Funciones}
La ultima manera de definir una función es usando una función flecha:
\begin{lstlisting}[language=JavaScript]
const sumNumbers = (num1, num2) => { 
    let total = num1 + num2; 
    return total;
}\end{lstlisting} \pause

Da igual como definas la función, al final todas se emplean de la misma manera:
\begin{lstlisting}[language=JavaScript]
sumNumbers(4, 8); // 12
\end{lstlisting}
\end{frame}

\begin{frame}[fragile]{Funciones: Argumentos}
Como dije antes, una función puede no tener argumentos de entrada ni de salida.

Si una función no tiene ningún argumento de salida, la función devuelve undefined por defecto.
\begin{lstlisting}[language=JavaScript]
function alertMe(){
    alert("ALERTA!");
}

alertMe(); // "undefined"
\end{lstlisting}
\end{frame}

\begin{frame}[fragile]{Funciones: Argumentos}
Si pasas mas argumentos a una función de los que necesita, esta ignorara los restantes.
\begin{lstlisting}[language=JavaScript]
function multiply(num1, num2){
    return num1 * num2;
}

producto(9, 2, "patata"); // 18
\end{lstlisting} \pause

Si pasas menos argumentos a una función de los que necesita, estos valen undefined por defecto.
\begin{lstlisting}[language=JavaScript]
function multiply(num1, num2){
    return num1 * num2;
}

producto(9); // 9 * undefined = "NaN"
\end{lstlisting}
\end{frame}

\begin{frame}[fragile]{Funciones: Argumentos}
En caso de que queráis tratar con un numero variable de argumentos, el objeto {\verb|arguments|} contiene todos los parámetros pasados a una función, así como su orden.

\begin{lstlisting}[language=JavaScript]
function dummy(){
    for (let argument of arguments){
        console.log(`got arg ${argument}`);
    }
}

dummy(3, "bomb", false, [1, "cat"]);
\end{lstlisting}
\end{frame}

\begin{frame}{Web Dev 101}
Antes de aplicar lo que hemos aprendido de JavaScript, voy a introduciros al mundo del desarrollo web. \pause

El desarrollo web se puede dividir en dos ramas principales: backend (servidor) y frontend (cliente). \pause

JavaScript puede ser utilizado tanto en backend como en frontend, pero cuando la gente piensa en una pagina web, suele pensar en el frontend.
    
\end{frame}

\begin{frame}{Web Dev 101}
Dentro del frontend, hay tres tecnologías básicas en las que toda pagina web se basa: HTML, CSS y JavaScript. \pause
\begin{itemize}
    \item HTML (Hyper Text Markdown Language) es el esqueleto de una pagina web. Tanto CSS como JavaScript se basan en dicho esqueleto y lo aumentan. \pause
    \item CSS (Cascade Style Sheets) es la piel de una pagina web. Su funcionalidad es cambiar el aspecto visual de una pagina web. \pause
    \item JavaScript son los músculos de una pagina web. Provee al esqueleto de funcionalidad, y sin el solo seria un cuerpo inerte.
\end{itemize}
\end{frame}

\begin{frame}{HTML}
\begin{figure}
    \centering
    \includegraphics[width=\textwidth]{images/nocss.png}
    \caption{Esta web es HTML puro.}
\end{figure}
\end{frame}

\begin{frame}{CSS}
\begin{figure}
    \centering
    \includegraphics[width=\textwidth]{images/css.png}
    \caption{Esta es la misma web, pero con CSS añadido.}
\end{figure}
\end{frame}

\begin{frame}{HTML en 3 minutos}
HTML \textbf{no} es un lenguaje de programación, ya que no realiza ningún tipo de lógica por detras.

HTML es un lenguaje de markup basado en tags (etiquetas), y cada etiqueta tiene una funcionalidad distinta. Este es un hello world:
    
\end{frame}

\begin{frame}[fragile]{HTML en 3 minutos}
\begin{lstlisting}[language=HTML]
<!DOCTYPE html>
<html>
    <head>
        <title>Example website</title>
    </head>

    <body>
        <h1>Hello HoC 18!</h1>
    </body>
</html> 
\end{lstlisting}
\end{frame}

\begin{frame}{HTML en 3 minutos}
\begin{figure}
    \centering
    \includegraphics[width=\textwidth]{images/helloworldhtml.png}
\end{figure}
\end{frame}

\begin{frame}[fragile]{CSS en 3 minutos}
CSS es un lenguaje que aplica estilos a los distintos elementos de una pagina HTML.
\begin{lstlisting}
body {
    background-color: red;
}

h1 {
    text-align: center;
    color: aqua;
    font-size: 10em;
}
\end{lstlisting}
\end{frame}


\begin{frame}{CSS en 3 minutos}
\begin{figure}
    \centering
    \includegraphics[width=\textwidth]{images/helloworldcss.png}
\end{figure}
\end{frame}

\begin{frame}{Combinando todo}
    Ahora que entendéis un poco el concepto de como funciona una pagina web, vamos a hacer un pequeño proyecto que combina todo.
    
    \centering\textbf{\hyperlink{https://jsbin.com/goqarifete/1/edit?js,console,output}{\Large jsbin.com/goqarifete?js,console,output}}
\end{frame}

\begin{frame}[fragile]{Diagrama de aprendizaje web}
Aquí tenéis un diagrama de que tecnologías aprender si queréis introduciros al desarrollo web.

\begin{tikzcd}[column sep=tiny]
HTML \arrow[d]                                          & Responsive Design                         & JQuery \arrow[d]\\
CSS \arrow[d] \arrow[ur, "upgrade"]                     & Frontend \arrow[ur] \arrow[r] & EscogeUnFramework \arrow[ld] \arrow[d] \arrow[rd]\\
JavaScript \arrow[ur] \arrow[dr] \arrow[d, "upgrade"]    & React                                          & Angular                                   & Vue\\ 
Asyncronous JS                                          & Backend \arrow[r]                         & Node.js\\
\end{tikzcd}
\end{frame}

\begin{frame}{TypeScript}
TypeScript es un superset de JavaScript (por lo que todo código de JavaScript funciona en TypeScript), el cual añade múltiples mejoras a este, como:
\begin{itemize}
    \item Tipado estático opcional
    \item Clases
    \item Interfaces
    \item Mejor integración en IDEs
\end{itemize}
\end{frame}

\begin{frame}[fragile]{TypeScript: Tipado estático}
Tenemos este codigo en JavaScript:
\begin{lstlisting}[language=JavaScript]
function sum(num1, num2){
    return num1 + num2;
}

sum(4,9); // 13
sum("potato", false) // "potatofalse"
\end{lstlisting}
\end{frame}

\begin{frame}[fragile]{TypeScript: Tipado estático}
En TypeScript, puedes hacer esto:
\begin{lstlisting}[language=JavaScript]
function sum(num1: number, num2: number){
    return num1 + num2;
}

sum(4,9); // 13
sum("potato", false) // error TS2345: Argument of type 'string' is not assignable to parameter of type 'number'.
\end{lstlisting}
\end{frame}

\begin{frame}[fragile]{Typescript: Clases}
TypeScript también introduce clases en JavaScript, las cuales son como un prototipo de objeto.

\begin{lstlisting}[language=JavaScript]
class Person {
    name: string;
    age: number;
    gender: string;
    }

function greeting(person: Person){
    console.log(`Hello, ${person.name}. You are ${person.gender} and ${person.age} years old.`);
}

myself = new Person("Carlos Gomez", 21, "male");
greeting(myself);
\end{lstlisting}
\end{frame}

\begin{frame}{Eso es todo}
Y eso es todo por esta presentación. 

Al final, esto es solo una introducción muy condensada al lenguaje, por lo que os he dejado varios links útiles en el README del repositorio.

Si tenéis cualquier duda o problema (O queréis tomar un café), soy @Kurolox en todos lados (GitHub, telegram...)
\end{frame}
\end{document}